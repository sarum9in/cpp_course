\include{header}

\begin{document}
    \title[C++]{C++}
\author[Филиппов]{А.~Филиппов\inst{1}}
\institute
{
    \inst{1}
    Ижевский государственный технический университет имени М.Т.~Калашникова
}
\frame{\titlepage}
\begin{frame}{Содержание}
    \tableofcontents
\end{frame}

    \begin{frame}{Фазы трансляции}
        \begin{itemize}
            \item Каждый комментарий заменяется одним пробельным символом.
            \item Процесс выделения токенов препроцессора контекстно-зависимый (например, символ в токенах препроцессора).
            \item Выполняются директивы препроцессора, разворачиваются макро-определения, выполняются унарные операторы \_Pragma.
                В случае, если в результате соединения токенов получается universal-character-name, \textit{поведение не определено}.
            \item Директивы $"\sharp{include}"$ запускают рекурсивную обработку указанного файла рекурсивно.
            \item Удаляются все директивы препроцессора.
            \item Происходит синтаксический и семантический анализ.
            \item (Опционально) происходит компановка единиц трансляции в исполняемые файлы либо статические библиотеки.
        \end{itemize}
    \end{frame}
    \section{Самостоятельно}
    \begin{frame}{Самостоятельно}
        \begin{itemize}
            \item Разобрать процесс трансляции для любого (выбранного) компилятора C++.
            \item Рассмотреть типы директив препроцессора.
        \end{itemize}
    \end{frame}
    \section{GCC}
    \subsection{Препроцессор}
    \subsubsection{Директивы}
    \subsection{Компилятор}
    \subsubsection{Ключевые слова}
    \subsection{Ассемблер}
    \subsubsection{Объектный файл}
    \subsubsection{Статическая библиотека}
    \subsection{Линкер}
    \subsubsection{Исполняемый файл}
    \subsubsection{Динамическая библиотека}

    %\section{Время жизни объекта}
    % 3.10 value categories
    %\subsection{}
    %\begin{frame}{}
    %\end{frame}
\end{document}
