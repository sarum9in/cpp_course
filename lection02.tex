\include{header}

\begin{document}
    \title[C++]{C++}
\author[Филиппов]{А.~Филиппов\inst{1}}
\institute
{
    \inst{1}
    Ижевский государственный технический университет имени М.Т.~Калашникова
}
\frame{\titlepage}
\begin{frame}{Содержание}
    \tableofcontents
\end{frame}

    \section{Сборка}
    \subsection{Раздельная компиляция}
    \subsubsection{Исходный код}
    \begin{frame}{Исходный код}
        \begin{itemize}
            \item Исходный код содержится в единицах, называемыми исходными файлами.
            \item Исходный файл вместе со всеми включенными исходными файлами и заголовочными файлами посредством директивы \#include,
                за исключением строк, пропущенных директивами условного включения, называется единицей трансляции.
        \end{itemize}
    \end{frame}
    \subsection{Фазы трансляции}
    \begin{frame}{Фазы трансляции}
        \begin{itemize}
            \item Реальные символы исходного файла отображаются в базовый набор символов.
                Множество принимаемых символов исходного кода \textit{задаётся реализацией языка}.
            \item Все символы вне базового набора заменяются на universal-character-name $\backslash{uXXXX}$.
            \item Все вхождения обратного слеша, сразу за которым следует символ перевода строки,
                удаляются, а соответствующие строки соединяются в одну. Если в результате появляется
                universal-character-name, \textit{поведение не определено}.
            \item Исходный файл разбивается на токены препроцессора и последовательности
                пробельных символов, включая комментарии. Файл не должен содержать не закрытых комментариев и
                токенов препроцессора.
            \item Подряд идущие строковые литералы конкатенируются.
        \end{itemize}
    \end{frame}
    \begin{frame}{Фазы трансляции}
        \begin{itemize}
            \item Каждый комментарий заменяется одним пробельным символом.
            \item Процесс выделения токенов препроцессора контекстно-зависимый (например, символ в токенах препроцессора).
            \item Выполняются директивы препроцессора, разворачиваются макро-определения, выполняются унарные операторы \_Pragma.
                В случае, если в результате соединения токенов получается universal-character-name, \textit{поведение не определено}.
            \item Директивы \#include запускают обработку указанного файла рекурсивно.
            \item Удаляются все директивы препроцессора.
            \item Происходит синтаксический и семантический анализ.
            \item (Опционально) происходит компоновка единиц трансляции в исполняемые файлы либо статические библиотеки.
        \end{itemize}
    \end{frame}
    \section{Комментарии}
    \begin{frame}{Блочные комментарии}
        \begin{itemize}
            \item \textbf{Синтаксис:} /* ... */.
            \item Не могут вкладываться.
            \item // и /* внутри такого комментария не имеют специального значения.
        \end{itemize}
    \end{frame}
    \begin{frame}{Строчные комментарии}
        \begin{itemize}
            \item \textbf{Синтаксис:} // ...\textbackslash{n}.
            \item Если внутри комментария встречается form-feed или vertical-tab, то после них могут идти только пробельные символы (в конце перевод строки).
            \item /*, */ и // внутри такого комментария не имеют специального значения.
        \end{itemize}
    \end{frame}
    \section{Препроцессор}
    \subsection{Директива}
    \begin{frame}{Директива}
        \begin{itemize}
            \item \textbf{Пример:} \#DIRECTIVE ARG1 ARG2...
            \item Описание (в данном контексте в пробельные символы не входит перенос строки)
                \begin{enumerate}
                    \item произвольное количество пробельных символов
                    \item \#
                    \item произвольное количество пробельных символов
                    \item имя директивы
                    \item произвольное ненулевое количество пробельных символов
                    \item тело директивы (с учётом экранированного переноса строки)
                    \item перенос строки
                \end{enumerate}
            \item Макросы внутри директив не разворачиваются если не сказано обратное.
        \end{itemize}
    \end{frame}
    \subsection{Условное включение}
    \subsubsection{Пример}
    \begin{frame}{Условное включение, пример}
        \lstinputlisting{examples/conditional_inclusion_simple.cpp}
    \end{frame}
    \subsubsection{Описание}
    \begin{frame}{Условное включение}
        \begin{itemize}
            \item Использует одно или несколько выражений, используемых для определения "включать ли код?".
            \item Выражения являются целочисленными, вычисляемыми на этапе препроцессерирования.
            \item Обработка данных происходит в std::intmax\_t или std::uintmax\_t.
            \item Специальным образом обрабатывается унарный оператор defined в форме "defined(symbol)" или "defined symbol",
                возвращающий 1 или 0 в зависимости от того, определён ли в данный момент "symbol" директивой \#define (без последующей \#undef).
        \end{itemize}
    \end{frame}
    \begin{frame}{Условное включение}
        \begin{itemize}
            \item В выражении происходит подстановка макроопределений.
            \item Если в результате подстановки макроопределений появится оператор defined, \textit{поведение не определено}.
            \item После подстановки и выполнения оператора defined все оставшиеся слова кроме true и false заменяются на 0.
            \item В выражении обрабатываются литералы типа char.
            \item Коды символов в препроцессоре могут отличаться от тех, что используются вне контекста препроцессора, поведение \textit{implementation-defined}.
            \item Директивы могут вкладываться.
        \end{itemize}
    \end{frame}
    \begin{frame}{Условное включение: синтаксис}
        \begin{enumerate}
            \item
                \begin{itemize}
                    \item \#if выражение
                    \item \#ifdef символ // эквивалентно \#if defined(символ)
                    \item \#ifndef символ // эквивалентно \#if !defined(символ)
                \end{itemize}
            \item произвольное количество \#elif выражение
            \item опционально \#else-директива
            \item \#endif
        \end{enumerate}
    \end{frame}
    \subsubsection{Примеры}
    \begin{frame}{Примеры}
        \lstinputlisting{examples/conditional_inclusion_if.cpp}
    \end{frame}
    \begin{frame}{Примеры}
        \lstinputlisting{examples/conditional_inclusion_ifdef.cpp}
    \end{frame}
    \begin{frame}{Примеры}
        \lstinputlisting{examples/conditional_inclusion_ifndef.cpp}
    \end{frame}
    \begin{frame}{Примеры}
        \lstinputlisting{examples/conditional_inclusion_wordsize.cpp}
    \end{frame}
    \begin{frame}{Примеры: что выведет код на экране?}
        \lstinputlisting{examples/conditional_inclusion_task.cpp}
    \end{frame}
    \subsection{Включение файлов с исходным кодом}
    \subsubsection{Описание}
    \begin{frame}{Включение файлов с исходным кодом}
        \begin{itemize}
            \item \#include <name> ищет файл "name" \textit{implementation-defined}.
            \item \#include "name" ищет файл "name" \textit{implementation-defined}, в случае неудачи,
                поиск происходит так, как будто использовался предыдущий вариант директивы \#include <name>.
            \item \#include TOKEN0 TOKEN1... производит подстановку макросов TOKEN, после чего выполняется директива.
        \end{itemize}
    \end{frame}
    \subsubsection{Примеры}
    \begin{frame}{Примеры}
        \lstinputlisting{examples/source_inclusion.cpp}
    \end{frame}
    \section{Самостоятельно}
    \begin{frame}{Самостоятельно}
        \begin{itemize}
            \item Разобрать процесс трансляции для любого (выбранного) компилятора C++.
            \item Рассмотреть типы директив препроцессора.
            \item Триграфы.
                \begin{itemize}
                    \item Что такое "триграф"?
                    \item Для чего они нужны?
                    \item Как они обрабатываются современными компиляторами?
                    \item На какой фазе трансляции обрабатываются триграфы?
                \end{itemize}
        \end{itemize}
    \end{frame}
\end{document}
