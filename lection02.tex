\documentclass[xetex,mathserif,serif,10pt]{beamer}
%\documentclass[11pt]{article}
\usepackage{xltxtra}
\usepackage{polyglossia}
\setdefaultlanguage[spelling=modern]{russian}
%\setmainfont[Mapping=tex-text]{DejaVu Sans}
%\setmainfont[Mapping=tex-text]{Liberation Sans}
\setmonofont[Mapping=tex-text]{DejaVu Sans Mono}
\setmainfont[Mapping=tex-text]{Linux Libertine O}
%\setmonofont[Mapping=tex-text]{Liberation Mono}

\usepackage{verbatim}
\usepackage{tabularx}
\usepackage{float}
\usepackage{url}

\usepackage{indentfirst}

\usepackage{algorithm}
\usepackage{algorithmic}

%\usepackage[a4paper]{geometry}
%\geometry{left=25mm}
%\geometry{right=25mm}
%\geometry{top=25mm}
%\geometry{bottom=15mm}

%\setbeamertemplate{caption}[numbered]

\newcommand{\includepicture}[3]{
\begin{figure}[H]
\begin{center}
\leavevmode
%\large{\textbf{#2}}
\includegraphics[#3]{#1}
\end{center}
\caption{#2}
\end{figure}
}

\usepackage{listings}
\lstset{ %
language=C++,                   % the language of the code
basicstyle=\scriptsize,         % the size of the fonts that are used for the code
numbers=left,                   % where to put the line-numbers
numberstyle=\footnotesize,      % the size of the fonts that are used for the line-numbers
stepnumber=1,                   % the step between two line-numbers. If it's 1, each line 
                                % will be numbered
numbersep=5pt,                  % how far the line-numbers are from the code
%backgroundcolor=\color{white},  % choose the background color. You must add \usepackage{color}
showspaces=false,               % show spaces adding particular underscores
showstringspaces=false,         % underline spaces within strings
showtabs=false,                 % show tabs within strings adding particular underscores
frame=single,                   % adds a frame around the code
tabsize=2,                      % sets default tabsize to 2 spaces
captionpos=b,                   % sets the caption-position to bottom
breaklines=true,                % sets automatic line breaking
breakatwhitespace=false,        % sets if automatic breaks should only happen at whitespace
title=\lstname,                 % show the filename of files included with \lstinputlisting;
                                % also try caption instead of title
escapeinside={\%*}{*)},         % if you want to add a comment within your code
morekeywords={*,...}            % if you want to add more keywords to the set
}

%\usetheme{Warsaw}
\usetheme{Singapore}


\begin{document}
    \title[C++]{C++}
\author[Филиппов]{А.~Филиппов\inst{1}}
\institute
{
    \inst{1}
    Ижевский государственный технический университет имени М.Т.~Калашникова
}
\frame{\titlepage}
\begin{frame}{Содержание}
    \tableofcontents
\end{frame}

    \section{Сборка}
    \subsection{Раздельная компиляция}
    \subsubsection{Исходный код}
    \begin{frame}{Исходный код}
        \begin{itemize}
            \item Исходный код содержится в единицах, называемыми исходными файлами.
            \item Исходный файл вместе со всеми включенными исходными файлами и заголовочными файлами посредством директивы \#include
                за исключением строк, пропущенным директивами условным включением, называется единица трансляции.
        \end{itemize}
    \end{frame}
    \subsection{Фазы трансляции}
    \begin{frame}{Фазы трансляции}
        \begin{itemize}
            \item Реальные символы исходного файла отображаются в базовый набор символов.
                Множество принимаемых символов исходного кода \textit{задаётся реализацией языка}.
            \item Все символы вне базового набора заменяются на universal-character-name $\backslash{uXXXX}$.
            \item Все вхождения обратного слеша, сразу за которым следует символ перевода строки,
                удаляются, а соответствующие строки соединяются в одну. Если в результате появляется
                universal-character-name, \textit{поведение не определено}.
            \item Исходный файл разбивается на токены препроцессора и последовательности
                пробельных символов, включая комментарии. Файл не должен содержать не закрытых комментариев и
                токенов препроцессора.
            \item Подряд идущие строковые литералы конкатенируются.
        \end{itemize}
    \end{frame}
    \begin{frame}{Фазы трансляции}
        \begin{itemize}
            \item Каждый комментарий заменяется одним пробельным символом.
            \item Процесс выделения токенов препроцессора контекстно-зависимый (например, символ в токенах препроцессора).
            \item Выполняются директивы препроцессора, разворачиваются макро-определения, выполняются унарные операторы \_Pragma.
                В случае, если в результате соединения токенов получается universal-character-name, \textit{поведение не определено}.
            \item Директивы \#include запускают рекурсивную обработку указанного файла рекурсивно.
            \item Удаляются все директивы препроцессора.
            \item Происходит синтаксический и семантический анализ.
            \item (Опционально) происходит компановка единиц трансляции в исполняемые файлы либо статические библиотеки.
        \end{itemize}
    \end{frame}
    \section{Комментарии}
    \begin{frame}{Блочные комментарии}
        \begin{itemize}
            \item \textbf{Синтаксис:} /* ... */.
            \item Не могут вкладываться.
            \item // и /* внутри такого комментария не имеют специального значения.
        \end{itemize}
    \end{frame}
    \begin{frame}{Строчные комментарии}
        \begin{itemize}
            \item \textbf{Синтаксис:} // ...\textbackslash{n}=.
            \item Если внутри комментария встречается form-feed или vertical-tab, то после них могут идти только пробельные символы (в конце перевод строки).
            \item /*, */ и // внутри такого комментария не имеют специального значения.
        \end{itemize}
    \end{frame}
    \section{Препроцессор}
    \subsection{Директива}
    \begin{frame}{Директива}
        \begin{itemize}
            \item \textbf{Пример:} \#DIRECTIVE ARG1 ARG2...
            \item Описание (в пробельные символы не входит перенос строки)
                \begin{enumerate}
                    \item произвольное количество пробельных символов
                    \item \#
                    \item произвольное количество пробельных символов
                    \item имя директивы
                    \item произвольное ненулевое количество пробельных символов
                    \item тело директивы (с учётом экранированного переноса строки)
                    \item перенос строки
                \end{enumerate}
            \item Макросы внутри директив не разворачиваются если не сказано обратное.
        \end{itemize}
    \end{frame}
    \subsection{Условное включение}
    \begin{frame}{Условное включение}
        \begin{itemize}
            \item
        \end{itemize}
    \end{frame}
    \subsection{Включение файлов с исходным кодом}
    \begin{frame}{Включение файлов с исходным кодом}
        \begin{itemize}
            \item \#include <name> ищет файл "name" \textit{implementation-defined}.
            \item \#include "name" ищет файл "name" \textit{implementation-defined}, в случае неудачи,
                поиск происходит так, как будто использовался предыдущий вариант директивы \#{include} <name>.
            \item \#{include} NAME
        \end{itemize}
    \end{frame}
    \section{Самостоятельно}
    \begin{frame}{Самостоятельно}
        \begin{itemize}
            \item Разобрать процесс трансляции для любого (выбранного) компилятора C++.
            \item Рассмотреть типы директив препроцессора.
            \item Триграфы.
                \begin{itemize}
                    \item Что такое "триграф"?
                    \item Для чего они нужны?
                    \item Как они обрабатываются современными компиляторами?
                    \item На какой фазе трансляции обрабатываются триграфы?
                \end{itemize}
        \end{itemize}
    \end{frame}
    \section{GCC}
    \subsection{Препроцессор}
    \subsubsection{Директивы}
    \subsection{Компилятор}
    \subsubsection{Ключевые слова}
    \subsection{Ассемблер}
    \subsubsection{Объектный файл}
    \subsubsection{Статическая библиотека}
    \subsection{Линкер}
    \subsubsection{Исполняемый файл}
    \subsubsection{Динамическая библиотека}

    % TODO operation sequence
    % TODO 1.10 threads
    % TODO signals


    %\section{Время жизни объекта}
    % 3.10 value categories
    %\subsection{}
    %\begin{frame}{}
    %\end{frame}
\end{document}
