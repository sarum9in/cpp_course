\documentclass[xetex,mathserif,serif,10pt]{beamer}
%\documentclass[11pt]{article}
\usepackage{xltxtra}
\usepackage{polyglossia}
\setdefaultlanguage[spelling=modern]{russian}
%\setmainfont[Mapping=tex-text]{DejaVu Sans}
%\setmainfont[Mapping=tex-text]{Liberation Sans}
\setmonofont[Mapping=tex-text]{DejaVu Sans Mono}
\setmainfont[Mapping=tex-text]{Linux Libertine O}
%\setmonofont[Mapping=tex-text]{Liberation Mono}

\usepackage{verbatim}
\usepackage{tabularx}
\usepackage{float}
\usepackage{url}

\usepackage{indentfirst}

\usepackage{algorithm}
\usepackage{algorithmic}

%\usepackage[a4paper]{geometry}
%\geometry{left=25mm}
%\geometry{right=25mm}
%\geometry{top=25mm}
%\geometry{bottom=15mm}

%\setbeamertemplate{caption}[numbered]

\newcommand{\includepicture}[3]{
\begin{figure}[H]
\begin{center}
\leavevmode
%\large{\textbf{#2}}
\includegraphics[#3]{#1}
\end{center}
\caption{#2}
\end{figure}
}

\usepackage{listings}
\lstset{ %
language=C++,                   % the language of the code
basicstyle=\scriptsize,         % the size of the fonts that are used for the code
numbers=left,                   % where to put the line-numbers
numberstyle=\footnotesize,      % the size of the fonts that are used for the line-numbers
stepnumber=1,                   % the step between two line-numbers. If it's 1, each line 
                                % will be numbered
numbersep=5pt,                  % how far the line-numbers are from the code
%backgroundcolor=\color{white},  % choose the background color. You must add \usepackage{color}
showspaces=false,               % show spaces adding particular underscores
showstringspaces=false,         % underline spaces within strings
showtabs=false,                 % show tabs within strings adding particular underscores
frame=single,                   % adds a frame around the code
tabsize=2,                      % sets default tabsize to 2 spaces
captionpos=b,                   % sets the caption-position to bottom
breaklines=true,                % sets automatic line breaking
breakatwhitespace=false,        % sets if automatic breaks should only happen at whitespace
title=\lstname,                 % show the filename of files included with \lstinputlisting;
                                % also try caption instead of title
escapeinside={\%*}{*)},         % if you want to add a comment within your code
morekeywords={*,...}            % if you want to add more keywords to the set
}

%\usetheme{Warsaw}
\usetheme{Singapore}


\begin{document}
    \title[C++]{C++}
\author[Филиппов]{А.~Филиппов\inst{1}}
\institute
{
    \inst{1}
    Ижевский государственный технический университет имени М.Т.~Калашникова
}
\frame{\titlepage}
\begin{frame}{Содержание}
    \tableofcontents
\end{frame}


    \section{Макроподстановка}
    \subsection{Список подстановки (replacement-list)}
    \begin{frame}{Список подстановки (replacement-list)}
        \begin{itemize}
            \item Два списка подстановки считаются равными тогда и только тогда, когда
                \begin{itemize}
                    \item состоят из одинакового количества токенов препроцессора
                    \item идущих в одинаковом порядке
                    \item имеющих одинаковое написание
                    \item разделённых одинаковыми последовательностями пробельных символов
                \end{itemize}
            \item Любые последовательности пробельных символов считаются одинаковыми.
        \end{itemize}
    \end{frame}
    \subsection{Виды макросов}
    \subsubsection{Объектный макрос: синтаксис}
    \begin{frame}{Объектный макрос: синтаксис}
        \begin{itemize}
            \item \#define identifier replacement-list \textbackslash{n}
        \end{itemize}
    \end{frame}
    \subsubsection{Функциональный макрос: синтаксис}
    \begin{frame}{Функциональный макрос: синтаксис}
        \begin{itemize}
            \item \#define identifier(identifier-list) replacement-list \textbackslash{n}
            \item \#define identifier(...) replacement-list \textbackslash{n}
            \item \#define identifier(identifier-list, ...) replacement-list \textbackslash{n}
        \end{itemize}
    \end{frame}
    \subsection{Определение макроса}
    \begin{frame}{Определение макроса}
        \begin{itemize}
            \item Определённый объектный макрос может быть переопределён другой директивой \#define,
                только если replacement-list нового определения повторяет старое, иначе \textit{программа некорректна}.
            \item Определённый функциональный макрос может быть переопределён другой директивой \#define,
                только если replacement-list нового определения, а так же количество, порядок и запись параметров
                повторяют старое определение, иначе \textit{программа некорректна}.
        \end{itemize}
    \end{frame}
    \begin{frame}{Определение макроса}
        \begin{itemize}
            \item Идентификатор и список подстановки разделяются пробелом (потому "\#define X (a) a" определит объектный макрос).
            \item Идентификатор, следующий сразу за директивой \#define называется именем макроса. Пространство имён для всех макросов одно.
            \item Любые пробелы до и после списка подстановки не считаются частью списка подстановки или макроса.
            \item Символ \#, за которым следует идентификатор, встретившиеся в месте, где может начинаться директива препроцессора,
                не являются кандидатом для макроподстановки.
        \end{itemize}
    \end{frame}
    \subsection{Использование макроса}
    \begin{frame}{}
        \begin{itemize}
            \item
        \end{itemize}
    \end{frame}
    \subsection{Подстановка аргументов (функциональные макросы)}
    \begin{frame}{Подстановка аргументов (функциональные макросы)}
        \begin{itemize}
            \item Параметры, встретившиеся в списке подстановки, заменяются на аргументы,
                в которых разворачиваются все макросы. Исключением являются
                параметры, перед которыми стоят операторы \# или \#\#, а также те,
                после которых стоит оператор \#\#: внтури них подстановка макросов не выполняется.
            \item Идентификатор \_\_VA\_ARGS\_\_, встретившийся в списке подстановки,
                обрабатывается как параметр, заменяемый на вариативные (variadic) аргументы.
        \end{itemize}
    \end{frame}
    \subsection{Оператор \#}
    \begin{frame}{Оператор \#}
        \begin{itemize}
            \item Символьный строковый литерал -- строковый литерал без префикса.
            \item После каждого символа \# должен стоять параметр макроса.
            \item Последовательность символа \# и параметра заменяется
                на запись аргумента.
            \item Пробельные символы до первого токена препроцессора аргумента
                и после последнего игнорируются.
            \item Пробельные символы между токенами препроцессора аргумента
                заменяются на один пробел.
        \end{itemize}
    \end{frame}
    \begin{frame}{Оператор \#}
        \begin{itemize}
            \item Пустому аргументу соответствует строка "".
            \item Символы `"' и `\textbackslash' экранируются символом `\textbackslash'.
            \item Если результат подстановки не является корректным символьным строковым литералом,
                \textit{поведение не определено}.
            \item Порядок операторов \# и \#\# \textit{не определён}.
        \end{itemize}
    \end{frame}
    \subsection{Оператор \#\#}
    \begin{frame}{Оператор \#\#}
        \begin{itemize}
            \item \#\# не должен встречаться в начале или конце определения макроса.
            \item В функциональных макросах параметр, стоящий до или после \#\#,
                заменяется на соответствующую аргументу последовательность токенов препроцессора.
            \item Если аргумент пустой, то он заменяется на \textit{placemarker}.
            \item До начала следующего цикла обработки макросов все последовательности
                \#\# удаляются, а токены препроцессора слева и справа от оператора конкатенируются.
        \end{itemize}
    \end{frame}
    \begin{frame}{Оператор \#\#}
        \begin{itemize}
            \item Конкатенация токена препроцессора и \textit{placemarker}'а даёт тот же самый токен препроцессора.
            \item Конкатенация двух \textit{placemarker}'ов даёт один \textit{placemarker}.
            \item Если в результате получается некорректный токен препроцессора, \textit{поведение не определено}.
            \item Порядок выполнения операторов \#\# \textit{не определён}.
        \end{itemize}
    \end{frame}
    \subsubsection{Примеры}
    \begin{frame}{Примеры}
        \lstinputlisting{examples/hash_hash.cpp}
    \end{frame}
    \subsection{Повторное сканирование и подстановка}
    \begin{frame}{Повторное сканирование и подстановка}
        \begin{itemize}
            \item После подстановки всех параметров и выполнения всех операторов \# и \#\# все \textit{placemarker}'ы удаляются.
            \item После этого получившаяся последовательность токенов сканируется снова для поиска макросов для подстановки.
            \item Если некоторая подпоследовательность токенов препроцессора является результатом выполнения макроса,
                то данный макрос внутри данной подпоследовательности токенов больше не выполняется.
            \item Получившаяся последовательность токенов не обрабатывается как директива препроцессора,
                даже если похожа на таковую. Но в ней выполняются унарные операторы \_Pragma.
        \end{itemize}
    \end{frame}
    \subsection{Время жизни макроса}
    \begin{frame}{Время жизни макроса}
        \begin{itemize}
            \item Макрос действует до первой, соответствующей ему, директива \#undef, либо до конца файла.
            \item Макрос не имеет значения после фазы препроцессирования.
        \end{itemize}
    \end{frame}
    \subsubsection{Директива \#undef}
    \begin{frame}{Директива \#undef}
        \begin{itemize}
            \item \#undef identifier \textbackslash{n}
            \item Отменяет действие указанного макроса.
            \item Игнорируется, если identifier не является именем макроса.
        \end{itemize}
    \end{frame}
    \subsection{Примеры}
    \begin{frame}{Stringize}
        \lstinputlisting{examples/stringize.cpp}
    \end{frame}
    \begin{frame}{Stringize complete}
        \lstinputlisting{examples/stringize_complete.cpp}
    \end{frame}

    \section{Самостоятельно}
    \begin{frame}{Самостоятельно}
        \begin{itemize}
            \item Альтернативные токены (Alternative tokens).
                \begin{itemize}
                    \item Альтернативная форма записи операторов.
                \end{itemize}
            \item
        \end{itemize}
    \end{frame}

    % TODO macro replacement
    \section{GCC}
    \subsection{Препроцессор}
    \subsubsection{Директивы}
    \subsection{Компилятор}
    \subsubsection{Ключевые слова}
    \subsection{Ассемблер}
    \subsubsection{Объектный файл}
    \subsubsection{Статическая библиотека}
    \subsection{Линкер}
    \subsubsection{Исполняемый файл}
    \subsubsection{Динамическая библиотека}

    % TODO operation sequence
    % TODO 1.10 threads
    % TODO signals


    %\section{Время жизни объекта}
    % 3.10 value categories
    %\subsection{}
    %\begin{frame}{}
    %\end{frame}
\end{document}
